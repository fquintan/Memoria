\chapter{Evaluación Experimental}
En este capítulo se divide en dos secciones, en la primera se describen los experimentos realizados para medir la eficiencia y eficacia de los sistemas implementados. En la segunda parte se muestran los resultados obtenidos.

\section{Dataset y Experimentos}
Para probar el sistema se recopiló una colección de videos correspondientes a películas de dominio público. Las peliculas fueron descargadas de un canal de Youtube que se especializa en subir recolectar y subir películas de dominio público\footnote{https://www.youtube.com/user/BestPDMovies}. En total se descargaron 119 películas que corresponden a alrededor de 110 horas de video. El dataset contiene películas publicadas desde 1920 y hasta la década de 1970 cuya protección de derechos de autor expiró, además de algunas más recientes que renunciaron a tales derechos. Debido a esto el dataset presenta videos con un amplio rango de tamaños y calidad de video, además de contener tanto filmes a color como en blanco y negro. 


Para comparar la eficacia de los distintos descriptores se usó la versión distribuida de la aplicación. Esto debido que es la que cuenta con el preprocesamiento encargado de ajustar el video a los límites indicados por el usuario, sin esto los descriptores globales no pueden funcionar para videos de tamaño indeterminado, como en la base de datos recopilada.

Para realizar las pruebas se seleccionaron aleatoriamente 10 videos de la base de datos, y para cada uno se realizaron 4 consultas con cada descriptor, esto nos da un total de 120 consultas. Se estudió el comportamiento de los descriptores al variar la cantidad de objetos en la base de datos de referencia. Para aislar el efecto de la cantidad de películas en la base de datos se guardaron, para cada consulta, los descriptores enviados por el cliente, de esta forma al repetir la consulta con una base de datos mayor podemos estar seguro que cualquier cambio en la efectividad fue debido al cambio en la base de datos y no a cambios del descriptor mismo. Se repitieron las consultas variando la cantidad de películas en la base de datos entre 15 y 119. Para cada consulta se contaron la cantidad de resultados correctos e incorrectos.

Para probar la eficiencia del sistema distribuido en comparación con el centralizado se midió tanto el tiempo de respuesta total del sistema como la cantidad de datos enviada por el cliente. 

El tiempo de respuesta se midió como el intervalo de tiempo desde que el cliente empieza a enviar datos hasta que termina de recibir la respuesta del servidor. Resulta interesante desglosar este tiempo en etapas para saber que operaciones son más costosas para el sistema. Para esto se midió en el servidor el tiempo gastado en distintas operaciones. En el caso del sistema centralizado se midió el tiempo usado por la extracción de descriptores y por la búsqueda en la base de datos. Para el caso del sistema distribuido se midió el tiempo usado escribiendo los archivos necesarios por P-VCD y los descriptores, así como el tiempo de la búsqueda en la base de datos. Finalmente se calculó el tiempo usado transmitiendo datos suponiendo que:
\begin{equation*}
\displaystyle{Tiempo\ total\ de\ respuesta}  =  \displaystyle{Tiempo\ de\ transmisi\acute{o}n} + \displaystyle{Tiempo\ total\ del\ servidor} 
\end{equation*}

\section{Resultados}
A continuación se presentan los resultados de los experimentos realizados y su interpretación. 

\subsection{Eficacia}



\subsection{Eficiencia}


%\lipsum[50-60]