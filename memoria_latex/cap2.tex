\chapter{Conceptos}

En este capítulo se presentan conceptos teóricos necesarios para la comprensión del trabajo realizado en la memoria, así como la descripción de  programas y frameworks utilizados cuyo funcionamiento y restricciones influyeron en decisiones de diseño tomadas a lo largo del desarrollo de la memoria.

\section{Definiciones}
\begin{enumerate}
\item Imagen: 
\end{enumerate}

\section{Descriptores}
Un \emph{descriptor} de imagen es una forma representar dicha imagen por sus características (como su distribución de colores, orientación de bordes, textura, etc). Los descriptores son usualmente representados como vectores multidimensionales. Para compararlos es común usar una función de distancia, como por ejemplo la distancia euclideana entre vectores.
El descriptor de una imagen puede ser global o local dependiendo de si describe la imagen completa o solo sectores de ella.
\section{Búsqueda por similitud}
\subsection{Comparación de descriptores}
\subsection{Comparación de videos}
\section{Renderscript}
\section{P-VCD}
\section{Medidas de evaluación}

\lipsum[50-60]