\chapter{Introducción} \label{intro}

\section{Motivación} \label{motivacion}


Día a día crece la cantidad de contenido audiovisual disponible gracias en parte a la televisión, a la proliferación de medios de captura de video de bajo costo y a la creciente oferta de servicios de almacenamiento de video en internet como Youtube o Vimeo.
Sin embargo, gran parte de este contenido no está etiquetado, es decir, no presenta metadatos que faciliten su rápida identificación. 

Para buscar dentro de colecciones de video no etiquetadas comúnmente se usa el enfoque de búsqueda por similitud. Este tipo de búsqueda se basa en comparar (esto es, medir cuanto se parecen) dos videos. Para medir esta similitud se utilizan \emph{descriptores}, que son vectores que representan características del contenido un video y bajo los cuales las comparaciones están definidas mediante funciones de distancia.

La masificación de los \emph{smartphones} o teléfonos inteligentes ha aportado al aumento de la cantidad de videos, y por ende al aumento de la cantidad de contenido no etiquetado, sin embargo por otro lado pueden ofrecernos una solución para su identificación. Existen ejemplos en los que estos dispositivos móviles ayudan a encontrar contenido no etiquetado como es el caso de la aplicación \emph{shazam}. Esta aplicación le permite a sus usuarios identificar contenido de audio que no contiene los datos necesarios para su identificación inmediata, por ejemplo, canciones en una radio. Sin embargo no existen soluciones similares para el caso de contenido audiovisual -como películas o series de televisión- a pesar de que estos dispositivos móviles permiten grabar videos fácilmente, y recientemente han aumentado su poder de cómputo de manera tal que les hace posible realizar tareas previamente relegadas a computadores de escritorio. 

La presente memoria se enmarca dentro del siguiente problema: identificar la \emph{fuente} de un segmento de video a partir de su contenido utilizando una búsqueda por similitud. El objetivo que se plantea es que un usuario pueda grabar parte del video no etiquetado usando su celular y, a través de una búsqueda por similitud, identificar el video. 
Adicionalmente se desea probar diversas implementaciones del sistema para medir sus diferencias en términos de eficiencia y eficacia, donde eficiencia se refiere al uso de recursos como la red de datos y el tiempo de procesamiento, y eficacia a la habilidad para retornar resultados correctos. Para realizar este trabajo es necesario ser capaz de realizar procesamiento de imágenes y búsqueda en contenido multimedia para llevar a cabo búsquedas por similitud del contenido que se está analizando. También es necesario manejar conceptos de programación concurrente de forma de poder implementar los descriptores que, a su vez, permiten llevar a cabo la búsqueda. Finalmente es ncesario ser capaz de desarrollar aplicaciones móviles para dispositivos android.

\section{Objetivos} \label{objetivos}

El objetivo general de esta memoria es implementar un sistema de búsqueda de videos en donde un usuario graba un video corto con un dispositivo móvil, luego se comunica con un servidor de consultas que identifique la fuente del video y finalmente le comunique sus resultados de vuelta a la aplicación. Se propone implementar el cálculo de descriptores de video para la búsqueda en el dispositivo móvil y comparar su eficiencia y uso de recursos con la alternativa de enviar el video completo para ser procesado por el servidor. Además se desea implementar distintos tipos de descriptores y evaluar su eficacia para reconocer el video de búsqueda.

Para conseguir el objetivo principal de la memoria se pueden detallar varios objetivos específicos los cuales son expuestos a continuación y agrupados en tres partes: 
\subsubsection*{Aplicación Android}
\begin{itemize}
\item Implementar una aplicación Android capaz de capturar frames usando la cámara del dispositivo.
\item Implementar un módulo que reciba frames de la cámara y calcule descriptores de imágenes, el módulo debe ser capaz de calcular descriptores de Histograma de grises, Histograma de bordes y Distribución de colores.
\item Obtener resultados de la búsqueda ejecutada por el servidor y mostrárselos al usuario.
\end{itemize}
\subsubsection*{Servidor}
\begin{itemize}
\item Descargar películas de dominio público e indexarlas para crear una base de datos de videos sobre la cual se realizarán consultas.
\item Implementar un servidor que reciba descriptores de la aplicación móvil y ejecute una búsqueda de copias en base a ellos.
\item Comunicarle los resultados de la búsqueda de vuelta a la aplicación.
\end{itemize}
\subsubsection*{Experimentos}
\begin{itemize}
\item Comparar los descriptores implementados en términos de su eficacia para la búsqueda.
\item Implementar una versión alternativa del sistema que reciba videos completos de la aplicación móvil y realice tanto el cálculo de descriptores como la búsqueda en el servidor.
\item Comparar las dos versiones del sistema en términos de sus eficiencia para cuantificar las ganancias comparativas logradas al realizar el cálculo de descriptores en el dispositivo móvil. 
\end{itemize}

\section{Estructura de la memoria}
El capítulo~\ref{intro} contiene la seccion~\ref{motivacion} que presenta una breve introducción y motivación de los problemas relevantes para la memoria y la sección~\ref{objetivos} donde se detalla el objetivo principal y una lista de objetivos específicos de esta memoria. En el capítulo~\ref{conceptos} define conceptos útiles en~\ref{definiciones}, para luego detallar el funcionamiento de la detección de copias en~\ref{copias} y los descriptores de video en~\ref{descriptores}. En las siguientes secciones se describen los programas y herramientas usadas, en~\ref{android} se describe el sistema Android, mientras que en~\ref{img_proc} se discuten herramientas disponibles para procesamiento de imágenes en Android, en~\ref{renderscript} el framework RenderScript y en~\ref{pvcd} el programa P-VCD.
El capítulo~\ref{cap_descriptores} expone los descriptores escogidos para implementar en esta memoria, ``Histograma de grises por zona'' en~\ref{ghd}, ``Distribución de colores'' en~\ref{cld} e ``Histograma de bordes'' en~\ref{ehd}.
En el capítulo~\ref{implementacion} se detalla la implementación del sistema, se divide dos secciones para describir las alternativas de arquitectura del sistema, una centralizada~\ref{central} y otra distribuida~\ref{distr}.
El capítulo~\ref{evaluacion} describe los experimentos realizados para evaluar el sistema en~\ref{experimentos}, para luego mostrar y discutir sus resultados en~\ref{resultados}.
Finalmente el capítulo~\ref{cap_conclusiones} presenta conclusiones finales de la memoria en~\ref{conclusiones}, y lista posible trabajo futuro en~\ref{futuro}.

%\lipsum[1-3]
%\begin{defn}[ver \cite{KAR00}] Definición definitiva $$\frac{d}{dx}\int_a^xf(y)dy=f(x).$$\end{defn}