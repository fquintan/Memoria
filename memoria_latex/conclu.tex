%\begin{conclusion}
%	\lipsum[130-132]
%	\begin{figure}[!h]
%		\centering
%		\includegraphics[scale=.2]{imagenes/fcfm}
%		\caption{Logo de la Facultad}
%		\label{logofcfm}
%	\end{figure}
%	\lipsum[133-134]
%	\begin{table}[!h]
%		\centering
%		\begin{tabular}{|c||c|}
%			\hline
%			Campo 1& Campo 2\\\hline
%			Valor 1& Valor2\\\hline
%		\end{tabular}
%		\caption{Tabla 1}
%		\label{tabla:1}
%	\end{table}
%	\lipsum[135]
%\end{conclusion}

\chapter{Conclusiones}\label{cap_conclusiones}
En este capítulo se presentan conclusiones finales sobre el trabajo realizado en la memoria, así como propuestas de trabajo futuro y posibles extensiones a la memoria.

\section{Conclusiones}\label{conclusiones}

Se cumplió satisfactoriamente con los objetivos establecidos al inicio de la memoria, habiendo implementado un sistema de búsqueda de videos capaz de calcular descriptores usando un dispositivo móvil y enviarlos a un servidor que utiliza algoritmos de detección de copias para identificar el video de búsqueda.

\subsection*{Eficiencia del sistema}
%Calcular descriptores en el dispositivo móvil consiguió reducir el significativamente tanto el tiempo de búsqueda como el uso de datos -> la aplicación se puede usar en red movil
Entre las falencias que se vislumbraban en la versión centralizada del sistema estaba el alto uso de la red de datos del dispositivo móvil y la excesiva carga del servidor al realizar todos los cálculos necesarios para la búsqueda. Los resultados obtenidos de las pruebas confirman estas falencias y reafirman la necesidad de rediseñar el sistema para resolverlas. Por otro lado los resultados validan el diseño propuesto demostrando que la implementación del cálculo de descriptores en el dispositivo móvil reduce significativamente la cantidad de datos enviados por el cliente y la carga de trabajo del servidor. El envío de datos desde el cliente pasó de más de 6 megabytes en la versión centralizada del sistema a cerca de 20 kilobytes en la versión distribuida. La carga de trabajo del servidor también se vio reducida en la versión distribuida, al traspasar el cálculo de descriptores al dispositivo móvil se elimina una operación que representaba un tercio del trabajo total del servidor.

A pesar de la reducción en el tiempo de respuesta del sistema lograda al realizar la extracción de descriptores en el dispositivo móvil la operación más cara del sistema, la búsqueda en la base de datos, es demasiado lenta como para que el sistema sea comercialmente viable. Si se quiere continuar el desarrollo de la aplicación como producto comercial será necesario realizar optimizaciones en el servidor que permitan reducir este tiempo de búsqueda. Ya que el tiempo de búsqueda crece linealmente con la cantidad de objetos en la base de datos, una optimización posible es extraer menos descriptores por segundo de video, esto disminuye fácilmente el tamaño de la base de datos. Sin embargo es necesario comprobar el efecto que esto tendría en la eficacia del sistema.

\subsection*{RenderScript}
%Renderscript apaño para calcular descriptores sería interesante ver que más se puede hacer
% -> mencionar tambien desventajas: nula documentacion, curva de aprendizaje, etc
El framework RenderScript de Android fue esencial para la implementación de los descriptores en el dispositivo móvil ya que permitió hacer uso eficiente de los recursos del teléfono. El framework permite al programador abstraerse de la arquitectura específica donde correrá el código dado que el sistema operativo se hace cargo de distribuir el código de manera eficiente para cada dispositivo. Resulta interesante entonces analizar otros sistemas que podrían verse beneficiados al traspasar trabajo del servidor al cliente o si existen tareas comúnmente relegadas a ordenadores de escritorio por sus altos requerimientos computacionales que pudiesen ser implementados en dispositivos móviles. 
Sin embargo también es necesario mencionar los problemas encontrados con el uso de RenderScript durante el desarrollo de la memoria. El framework inicialmente fue desarrollado específicamente para su uso en computación gráfica y contaba con una API de rendereo de imágenes. Esta API fue deprecada pronto después de ser lanzada y el framework se relanzó como orientado a cualquier tipo de cálculo que requiera alto paralelismo. A pesar de haber sido lanzado junto con la API 4.2 de Android a principios de 2013 aún existe escasa documentación de su funcionalidad. La falta de información de parte del equipo desarrollador del framework hace que su futuro sea incierto, lo cual dificulta la decisión de utilizarlo para construir aplicaciones comerciales.

\subsection*{Eficacia de los descriptores}
%Los descriptores usados no apañaron tanto, que alternativas hay?
De los tres tipos de descriptores usados solo uno (KFD) mostró una efectividad satisfactoria mientras que los otros dos (GHD y EHD) mostraron resultados peores que el azar en los experimentos realizados. Es importante comprender las razones para el desempeño medido si se quiere mejorarlo en futuras versiones del sistema. Es posible por ejemplo analizar la distribución de cada tipo de descriptor en la base de datos de videos originales para confirmar si alguno presenta una distribución que naturalmente dificulte su diferenciación. Por otro lado se puede simular la transformación sufrida por el video original al ser grabado con la cámara del celular para investigar la resistencia de cada tipo de descriptor a esta transformación.

Finalmente también es posible implementar distintas estrategias para extraer descriptores del video. Por un lado se puede usar un enfoque de tipo \emph{bag of words} usando descriptores locales. Esta estrategia se basa en extraer descriptores locales para crear un vocabulario de palabras visuales de la imagen, luego cada imagen se describe con un histograma de las \emph{palabras} presentes en la imagen. Este tipo de estrategia se encuentra disponible en el programa P-VCD por lo que nuevamente solo sería necesario la implementación del cálculo de descriptores en Android.

Por otro lado se pueden seguir usando descriptores globales, pero automatizando el proceso de encuadrar el video. Actualmente la aplicación le pide al usuario encuadrar el video correctamente en el marco de la pantalla. Este proceso se puede automatizar realizando un procesamiento de la imagen que detecte los límites de la pantalla y que luego recorte la imagen para solo calcular el descriptor global dentro de estos límites. Al mejorar la detección de los límites de la pantalla los descriptores globales debiesen mejorar su efectividad.

\section{Trabajo Futuro}\label{futuro}

A continuación se presentan propuestas de trabajo futuro:

\begin{itemize}
\item Analizar la eficacia de los descriptores usados al variar parámetros del sistema (como cuantos frames por segundo se extraen) y parámetros de cada descriptor.
\item Analizar las razones del mal desempeño de los descriptores GHD y EHD, estudiando su distribución en la base de datos y su resistencia a las transformaciones del video provocadas por la cámara.
\item Implementar nuevas estrategias de extracción de descriptores, como el uso de descriptores locales o la detección automática del marco de la pantalla e implementarlas en el dispositivo móvil.
\item Estudiar otros sistemas que puedan verse beneficiados por la implementación de cálculos en dispositivos móviles.
\item Estudiar en detalle el desempeño y las límitaciones de RenderScript usando algoritmos paralelos conocidos.
\end{itemize}

