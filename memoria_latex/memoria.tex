\documentclass[upright, contnum]{umemoria}
\usepackage[spanish]{babel}
\usepackage[utf8]{inputenc}

\depto{DEPARTAMENTO DE CIENCIAS DE LA COMPUTACIÓN}
\author{FELIPE ANDRÉS QUINTANILLA MATEFF}
\title{CÁLCULO DE DESCRIPTORES DE IMÁGENES EN DISPOSITIVOS ANDROID PARA UN SERVICIO DE BÚSQUEDA DE VIDEOS}
\auspicio{}
\date{AGOSTO 2015}
\guia{BENJAMÍN BUSTOS CÁRDENAS}
\carrera{INGENIERO CIVIL EN COMPUTACIÓN}
\memoria{MEMORIA PARA OPTAR AL TÍTULO DE}
\comision{JAVIER BUSTOS JIMÉNEZ}{AIDAN HOGAN }{\ }

\usepackage{lipsum}


%\usepackage[latin1]{inputenc}
\usepackage[T1]{fontenc}

%\usepackage{minted}
%\usepackage{multirow}
%\colorlet{LightGray}{gray!5!}
\renewcommand{\lstlistingname}{Código}
\definecolor{mygray}{rgb}{0.95,0.95,0.95}
\definecolor{mygray2}{rgb}{0.99,0.99,0.79}

%Defino un par de estilos
\lstdefinestyle{CInputStyle}{
  language=C,
  basicstyle=\small\sffamily,
  numbers=left,
  numberstyle=\tiny,
  numbersep=3pt,
  frame=tb,
  columns=fullflexible,
  backgroundcolor=\color{mygray},
  linewidth=0.9\linewidth,
  xleftmargin=0.05\linewidth
}

\lstdefinestyle{BashInputStyle}{
  language=bash,
  basicstyle=\small\sffamily,
  frame=tb,
  columns=fullflexible,
  backgroundcolor=\color{mygray2}
}

\begin{document}

\frontmatter
\maketitle

\begin{abstract}
{\lipsum[1-4]}
\end{abstract}

\begin{dedicatoria} % opcional
Una dedicatoria corta. Por ejemplo, \emph{A los creadores de U-Campus}
\end{dedicatoria}

\begin{thanks} % opcional
\lipsum[1-2]
\end{thanks}
\cleardoublepage

\tableofcontents
\listoftables % opcional
\listoffigures % opcional

\mainmatter

\input{intro.tex}
\chapter{Introducción} \label{intro}

\section{Motivación} \label{motivacion}


Día a día crece la cantidad de contenido audiovisual disponible gracias en parte a la televisión, a la proliferación de medios de captura de video de bajo costo y a la creciente oferta de servicios de almacenamiento de video en internet como Youtube o Vimeo.
Sin embargo, gran parte de este contenido no está etiquetado, es decir, no presenta metadatos que faciliten su rápida identificación. 

Para buscar dentro de colecciones de video no etiquetadas comúnmente se usa el enfoque de búsqueda por similitud. Este tipo de búsqueda se basa en comparar (esto es, medir cuanto se parecen) dos videos. Para medir esta similitud se utilizan \emph{descriptores}, que son vectores que representan características del contenido un video y bajo los cuales las comparaciones están definidas mediante funciones de distancia.

La masificación de los \emph{smartphones} o teléfonos inteligentes ha aportado al aumento de la cantidad de videos, y por ende al aumento de la cantidad de contenido no etiquetado, sin embargo por otro lado pueden ofrecernos una solución para su identificación. Existen ejemplos en los que estos dispositivos móviles ayudan a encontrar contenido no etiquetado como es el caso de la aplicación \emph{shazam}. Esta aplicación le permite a sus usuarios identificar contenido de audio que no contiene los datos necesarios para su identificación inmediata, por ejemplo, canciones en una radio. Sin embargo no existen soluciones similares para el caso de contenido audiovisual -como películas o series de televisión- a pesar de que estos dispositivos móviles permiten grabar videos fácilmente, y recientemente han aumentado su poder de cómputo de manera tal que les hace posible realizar tareas previamente relegadas a computadores de escritorio. 

La presente memoria se enmarca dentro del siguiente problema: identificar la \emph{fuente} de un segmento de video a partir de su contenido utilizando una búsqueda por similitud. El objetivo que se plantea es que un usuario pueda grabar parte del video no etiquetado usando su celular y, a través de una búsqueda por similitud, identificar el video. 
Adicionalmente se desea probar diversas implementaciones del sistema para medir sus diferencias en términos de eficiencia y eficacia, donde eficiencia se refiere al uso de recursos como la red de datos y el tiempo de procesamiento, y eficacia a la habilidad para retornar resultados correctos. Para realizar este trabajo es necesario ser capaz de realizar procesamiento de imágenes y búsqueda en contenido multimedia para llevar a cabo búsquedas por similitud del contenido que se está analizando. También es necesario manejar conceptos de programación concurrente de forma de poder implementar los descriptores que, a su vez, permiten llevar a cabo la búsqueda. Finalmente es ncesario ser capaz de desarrollar aplicaciones móviles para dispositivos android.

\section{Objetivos} \label{objetivos}

El objetivo general de esta memoria es implementar un sistema de búsqueda de videos en donde un usuario graba un video corto con un dispositivo móvil, luego se comunica con un servidor de consultas que identifique la fuente del video y finalmente le comunique sus resultados de vuelta a la aplicación. Se propone implementar el cálculo de descriptores de video para la búsqueda en el dispositivo móvil y comparar su eficiencia y uso de recursos con la alternativa de enviar el video completo para ser procesado por el servidor. Además se desea implementar distintos tipos de descriptores y evaluar su eficacia para reconocer el video de búsqueda.

Para conseguir el objetivo principal de la memoria se pueden detallar varios objetivos específicos los cuales son expuestos a continuación y agrupados en tres partes: 
\subsubsection*{Aplicación Android}
\begin{itemize}
\item Implementar una aplicación Android capaz de capturar frames usando la cámara del dispositivo.
\item Implementar un módulo que reciba frames de la cámara y calcule descriptores de imágenes, el módulo debe ser capaz de calcular descriptores de Histograma de grises, Histograma de bordes y Distribución de colores.
\item Obtener resultados de la búsqueda ejecutada por el servidor y mostrárselos al usuario.
\end{itemize}
\subsubsection*{Servidor}
\begin{itemize}
\item Descargar películas de dominio público e indexarlas para crear una base de datos de videos sobre la cual se realizarán consultas.
\item Implementar un servidor que reciba descriptores de la aplicación móvil y ejecute una búsqueda de copias en base a ellos.
\item Comunicarle los resultados de la búsqueda de vuelta a la aplicación.
\end{itemize}
\subsubsection*{Experimentos}
\begin{itemize}
\item Comparar los descriptores implementados en términos de su eficacia para la búsqueda.
\item Implementar una versión alternativa del sistema que reciba videos completos de la aplicación móvil y realice tanto el cálculo de descriptores como la búsqueda en el servidor.
\item Comparar las dos versiones del sistema en términos de sus eficiencia para cuantificar las ganancias comparativas logradas al realizar el cálculo de descriptores en el dispositivo móvil. 
\end{itemize}

\section{Estructura de la memoria}
El capítulo~\ref{intro} contiene la seccion~\ref{motivacion} que presenta una breve introducción y motivación de los problemas relevantes para la memoria y la sección~\ref{objetivos} donde se detalla el objetivo principal y una lista de objetivos específicos de esta memoria. En el capítulo~\ref{conceptos} define conceptos útiles en~\ref{definiciones}, para luego detallar el funcionamiento de la detección de copias en~\ref{copias} y los descriptores de video en~\ref{descriptores}. En las siguientes secciones se describen los programas y herramientas usadas, en~\ref{android} se describe el sistema Android, mientras que en~\ref{img_proc} se discuten herramientas disponibles para procesamiento de imágenes en Android, en~\ref{renderscript} el framework RenderScript y en~\ref{pvcd} el programa P-VCD.
El capítulo~\ref{cap_descriptores} expone los descriptores escogidos para implementar en esta memoria, ``Histograma de grises por zona'' en~\ref{ghd}, ``Distribución de colores'' en~\ref{cld} e ``Histograma de bordes'' en~\ref{ehd}.
En el capítulo~\ref{implementacion} se detalla la implementación del sistema, se divide dos secciones para describir las alternativas de arquitectura del sistema, una centralizada~\ref{central} y otra distribuida~\ref{distr}.
El capítulo~\ref{evaluacion} describe los experimentos realizados para evaluar el sistema en~\ref{experimentos}, para luego mostrar y discutir sus resultados en~\ref{resultados}.
Finalmente el capítulo~\ref{cap_conclusiones} presenta conclusiones finales de la memoria en~\ref{conclusiones}, y lista posible trabajo futuro en~\ref{futuro}.

%\lipsum[1-3]
%\begin{defn}[ver \cite{KAR00}] Definición definitiva $$\frac{d}{dx}\int_a^xf(y)dy=f(x).$$\end{defn}
\chapter{Conceptos}

En este capítulo se presentan conceptos teóricos necesarios para la comprensión del trabajo realizado en la memoria, así como la descripción de  programas y frameworks utilizados cuyo funcionamiento y restricciones influyeron en decisiones de diseño tomadas a lo largo del desarrollo de la memoria.

\section{Definiciones}
\begin{enumerate}
\item Imagen: 
\end{enumerate}

\section{Descriptores}
Un \emph{descriptor} de imagen es una forma representar dicha imagen por sus características (como su distribución de colores, orientación de bordes, textura, etc). Los descriptores son usualmente representados como vectores multidimensionales. Para compararlos es común usar una función de distancia, como por ejemplo la distancia euclideana entre vectores.
El descriptor de una imagen puede ser global o local dependiendo de si describe la imagen completa o solo sectores de ella.
\section{Búsqueda por similitud}
\subsection{Comparación de descriptores}
\subsection{Comparación de videos}
\section{Renderscript}
\section{P-VCD}
\section{Medidas de evaluación}

\lipsum[50-60]
\chapter{Descriptores}
\section{Histograma de grises}
\section{Reduced Keyframe}
\section{Histograma de bordes}

\lipsum[50-60]
\chapter{Implementacion}
En el siguiente capítulo se describe la implementación del sistema de reconocimiento de videos
\section{Cliente Android}
\subsection{a}
\subsection{b}
\subsection{c}
\section{Servidor}

\lipsum[50-60]
\chapter{Evaluación Experimental}
En este capítulo se divide en dos secciones, en la primera se describen los experimentos realizados para medir la eficiencia y eficacia de los sistemas implementados. En la segunda parte se muestran los resultados obtenidos.

\section{Dataset y Experimentos}
Para probar el sistema se recopiló una colección de videos correspondientes a películas de dominio público. Las peliculas fueron descargadas de un canal de Youtube que se especializa en subir recolectar y subir películas de dominio público\footnote{https://www.youtube.com/user/BestPDMovies}. En total se descargaron 119 películas que corresponden a alrededor de 110 horas de video. El dataset contiene películas publicadas desde 1920 y hasta la década de 1970 cuya protección de derechos de autor expiró, además de algunas más recientes que renunciaron a tales derechos. Debido a esto el dataset presenta videos con un amplio rango de tamaños y calidad de video, además de contener tanto filmes a color como en blanco y negro. 


Para comparar la eficacia de los distintos descriptores se usó la versión distribuida de la aplicación. Esto debido que es la que cuenta con el preprocesamiento encargado de ajustar el video a los límites indicados por el usuario, sin esto los descriptores globales no pueden funcionar para videos de tamaño indeterminado, como en la base de datos recopilada.

Para realizar las pruebas se seleccionaron aleatoriamente 10 videos de la base de datos, y para cada uno se realizaron 4 consultas con cada descriptor, esto nos da un total de 120 consultas. Se estudió el comportamiento de los descriptores al variar la cantidad de objetos en la base de datos de referencia. Para aislar el efecto de la cantidad de películas en la base de datos se guardaron, para cada consulta, los descriptores enviados por el cliente, de esta forma al repetir la consulta con una base de datos mayor podemos estar seguro que cualquier cambio en la efectividad fue debido al cambio en la base de datos y no a cambios del descriptor mismo. Se repitieron las consultas variando la cantidad de películas en la base de datos entre 15 y 119. Para cada consulta se contaron la cantidad de resultados correctos e incorrectos.

Para probar la eficiencia del sistema distribuido en comparación con el centralizado se midió tanto el tiempo de respuesta total del sistema como la cantidad de datos enviada por el cliente. 

El tiempo de respuesta se midió como el intervalo de tiempo desde que el cliente empieza a enviar datos hasta que termina de recibir la respuesta del servidor. Resulta interesante desglosar este tiempo en etapas para saber que operaciones son más costosas para el sistema. Para esto se midió en el servidor el tiempo gastado en distintas operaciones. En el caso del sistema centralizado se midió el tiempo usado por la extracción de descriptores y por la búsqueda en la base de datos. Para el caso del sistema distribuido se midió el tiempo usado escribiendo los archivos necesarios por P-VCD y los descriptores, así como el tiempo de la búsqueda en la base de datos. Finalmente se calculó el tiempo usado transmitiendo datos suponiendo que:
\begin{equation*}
\displaystyle{Tiempo\ total\ de\ respuesta}  =  \displaystyle{Tiempo\ de\ transmisi\acute{o}n} + \displaystyle{Tiempo\ total\ del\ servidor} 
\end{equation*}

\section{Resultados}
A continuación se presentan los resultados de los experimentos realizados y su interpretación. 

\subsection{Eficacia}



\subsection{Eficiencia}


%\lipsum[50-60]
%\begin{conclusion}
%	\lipsum[130-132]
%	\begin{figure}[!h]
%		\centering
%		\includegraphics[scale=.2]{imagenes/fcfm}
%		\caption{Logo de la Facultad}
%		\label{logofcfm}
%	\end{figure}
%	\lipsum[133-134]
%	\begin{table}[!h]
%		\centering
%		\begin{tabular}{|c||c|}
%			\hline
%			Campo 1& Campo 2\\\hline
%			Valor 1& Valor2\\\hline
%		\end{tabular}
%		\caption{Tabla 1}
%		\label{tabla:1}
%	\end{table}
%	\lipsum[135]
%\end{conclusion}

\chapter{Conclusiones}\label{cap_conclusiones}
En este capítulo se presentan conclusiones finales sobre el trabajo realizado en la memoria, así como propuestas de trabajo futuro y posibles extensiones a la memoria.

\section{Conclusiones}\label{conclusiones}

Se cumplió satisfactoriamente con los objetivos establecidos al inicio de la memoria, habiendo implementado un sistema de búsqueda de videos capaz de calcular descriptores usando un dispositivo móvil y enviarlos a un servidor que utiliza algoritmos de detección de copias para identificar el video de búsqueda.

\subsection*{Eficiencia del sistema}
%Calcular descriptores en el dispositivo móvil consiguió reducir el significativamente tanto el tiempo de búsqueda como el uso de datos -> la aplicación se puede usar en red movil
Entre las falencias que se vislumbraban en la versión centralizada del sistema estaba el alto uso de la red de datos del dispositivo móvil y la excesiva carga del servidor al realizar todos los cálculos necesarios para la búsqueda. Los resultados obtenidos de las pruebas confirman estas falencias y reafirman la necesidad de rediseñar el sistema para resolverlas. Por otro lado los resultados validan el diseño propuesto demostrando que la implementación del cálculo de descriptores en el dispositivo móvil reduce significativamente la cantidad de datos enviados por el cliente y la carga de trabajo del servidor. El envío de datos desde el cliente pasó de más de 6 megabytes en la versión centralizada del sistema a cerca de 20 kilobytes en la versión distribuida. La carga de trabajo del servidor también se vio reducida en la versión distribuida, al traspasar el cálculo de descriptores al dispositivo móvil se elimina una operación que representaba un tercio del trabajo total del servidor.

A pesar de la reducción en el tiempo de respuesta del sistema lograda al realizar la extracción de descriptores en el dispositivo móvil la operación más cara del sistema, la búsqueda en la base de datos, es demasiado lenta como para que el sistema sea comercialmente viable. Si se quiere continuar el desarrollo de la aplicación como producto comercial será necesario realizar optimizaciones en el servidor que permitan reducir este tiempo de búsqueda. Ya que el tiempo de búsqueda crece linealmente con la cantidad de objetos en la base de datos, una optimización posible es extraer menos descriptores por segundo de video, esto disminuye fácilmente el tamaño de la base de datos. Sin embargo es necesario comprobar el efecto que esto tendría en la eficacia del sistema.

\subsection*{RenderScript}
%Renderscript apaño para calcular descriptores sería interesante ver que más se puede hacer
% -> mencionar tambien desventajas: nula documentacion, curva de aprendizaje, etc
El framework RenderScript de Android fue esencial para la implementación de los descriptores en el dispositivo móvil ya que permitió hacer uso eficiente de los recursos del teléfono. El framework permite al programador abstraerse de la arquitectura específica donde correrá el código dado que el sistema operativo se hace cargo de distribuir el código de manera eficiente para cada dispositivo. Resulta interesante entonces analizar otros sistemas que podrían verse beneficiados al traspasar trabajo del servidor al cliente o si existen tareas comúnmente relegadas a ordenadores de escritorio por sus altos requerimientos computacionales que pudiesen ser implementados en dispositivos móviles. 
Sin embargo también es necesario mencionar los problemas encontrados con el uso de RenderScript durante el desarrollo de la memoria. El framework inicialmente fue desarrollado específicamente para su uso en computación gráfica y contaba con una API de rendereo de imágenes. Esta API fue deprecada pronto después de ser lanzada y el framework se relanzó como orientado a cualquier tipo de cálculo que requiera alto paralelismo. A pesar de haber sido lanzado junto con la API 4.2 de Android a principios de 2013 aún existe escasa documentación de su funcionalidad. La falta de información de parte del equipo desarrollador del framework hace que su futuro sea incierto, lo cual dificulta la decisión de utilizarlo para construir aplicaciones comerciales.

\subsection*{Eficacia de los descriptores}
%Los descriptores usados no apañaron tanto, que alternativas hay?
De los tres tipos de descriptores usados solo uno (KFD) mostró una efectividad satisfactoria mientras que los otros dos (GHD y EHD) mostraron resultados peores que el azar en los experimentos realizados. Es importante comprender las razones para el desempeño medido si se quiere mejorarlo en futuras versiones del sistema. Es posible por ejemplo analizar la distribución de cada tipo de descriptor en la base de datos de videos originales para confirmar si alguno presenta una distribución que naturalmente dificulte su diferenciación. Por otro lado se puede simular la transformación sufrida por el video original al ser grabado con la cámara del celular para investigar la resistencia de cada tipo de descriptor a esta transformación.

Finalmente también es posible implementar distintas estrategias para extraer descriptores del video. Por un lado se puede usar un enfoque de tipo \emph{bag of words} usando descriptores locales. Esta estrategia se basa en extraer descriptores locales para crear un vocabulario de palabras visuales de la imagen, luego cada imagen se describe con un histograma de las \emph{palabras} presentes en la imagen. Este tipo de estrategia se encuentra disponible en el programa P-VCD por lo que nuevamente solo sería necesario la implementación del cálculo de descriptores en Android.

Por otro lado se pueden seguir usando descriptores globales, pero automatizando el proceso de encuadrar el video. Actualmente la aplicación le pide al usuario encuadrar el video correctamente en el marco de la pantalla. Este proceso se puede automatizar realizando un procesamiento de la imagen que detecte los límites de la pantalla y que luego recorte la imagen para solo calcular el descriptor global dentro de estos límites. Al mejorar la detección de los límites de la pantalla los descriptores globales debiesen mejorar su efectividad.

\section{Trabajo Futuro}\label{futuro}

A continuación se presentan propuestas de trabajo futuro:

\begin{itemize}
\item Analizar la eficacia de los descriptores usados al variar parámetros del sistema (como cuantos frames por segundo se extraen) y parámetros de cada descriptor.
\item Analizar las razones del mal desempeño de los descriptores GHD y EHD, estudiando su distribución en la base de datos y su resistencia a las transformaciones del video provocadas por la cámara.
\item Implementar nuevas estrategias de extracción de descriptores, como el uso de descriptores locales o la detección automática del marco de la pantalla e implementarlas en el dispositivo móvil.
\item Estudiar otros sistemas que puedan verse beneficiados por la implementación de cálculos en dispositivos móviles.
\item Estudiar en detalle el desempeño y las límitaciones de RenderScript usando algoritmos paralelos conocidos.
\end{itemize}



% \input{glosario.tex} % opcional

\bibliographystyle{plain}
\bibliography{bibliografia}

% \input{anexo_apendices.tex} % opcionales

\end{document}
