\chapter{Implementación}
En el siguiente capítulo se describe la implementación del sistema desarrollado durante la memoria. Se implementaron dos posibles arquitecturas para un sistema de reconocimiento de videos. La primera alternativa graba videos con el dispositivo móvil y los envía completos al servidor, para que este calcule sus descriptores y realice la búsqueda, llamamos a esta la alternativa \emph{centralizada} pues todos los cálculos son realizados en el servidor. La segunda alternativa del sistema realiza los cálculos de descriptores en el mismo dispositivo móvil que graba el video, luego solo es necesario enviar los descriptores calculados al servidor que ejecuta la búsqueda, llamamos a esta alternativa \emph{distribuida} pues los cálculos están repartidos entre el cliente y el servidor.

El capítulo se divide en una sección para la alternativa centralizada y otra para la distribuida. A su vez cada sección se divide en la implementación del cliente Android y el servidor.
\TODO{Pensar en mejores nombres para los sistemas alternativos}


\section{Sistema Centralizado}
\subsection{Cliente Android}
\subsection{Servidor}
\section{Sistema Distribuido}
En esta versión del sistema el cálculo de descriptores para la búsqueda es realizado por el cliente Android utilizando las herramientas para computación paralela descritas en secciones anteriores. Una vez calculados, se envían solo estos descriptores al servidor el cual los usa para realizar la búsqueda por similitud y luego comunica los resultados de vuelta al cliente como ilustra la Figura~\ref{imagenes/cap3/arquitectura_distribuida.png}. \TODO{agregar imagen de la arquitectura distribuida}

\subsection{Cliente Android}
\subsection{Servidor}
